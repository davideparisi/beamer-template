\section{Premessa}

\begin{frame}{Premessa (1)}
Una Rete di Petri (Petri Net - PN) è
\begin{itemize}
	\item un \alert{modello} astratto e formale
	\item per la rappresentazione del \alert{comportamento dinamico} di sistemi \alert{discreti}
	\item che esibiscono attività \alert{asincrone e concorrenti}
\end{itemize}
\end{frame}

\begin{frame}{Premessa (2)}
	Molto usate nella modellizzazione di:
	\begin{itemize}
		\item Sistemi concorrenti
		\item Interazione tra sistemi diversi, compresa utente-computer
		\item Protocolli di comunicazione
        \item Workflow
        \item Sistemi complessi
        \item \dots
	\end{itemize}
\end{frame}

\begin{frame}{Notazione (1)}
    Concettualmente una PN è costituita
    \begin{itemize}
        \item da un insieme di elementi, detti \alert{posti}, che rappresentano i possibili stati del sistema
    \end{itemize}
\end{frame}

\section{Petri Nets}

\begin{frame}{Esempio (1)}
\begin{center}
	\begin{tikzpicture}
		% Styles
		[
			every label/.style={red},
			place/.style={circle,draw=blue!50,fill=blue!20,thick,inner sep=0,minimum size=6mm},
			transition/.style={rectangle,draw=black!50,fill=black!20,thick,inner sep=0,minimum size=4mm},
			bend angle=45,
			entrata/.style={<-,shorten <=1pt,>=stealth',thick},
			uscita/.style={->,shorten >=1pt,>=stealth',thick}
		]
		\node 	[place]	[label=right:$p_0$,]					(p0)								{};
		\node	[transition] [label=right:$t_0$]				(t0)	[below=of p0]				{}
			edge [entrata] (p0);
		\node	[place,tokens=1] [label=below:$p_1$]			(p1)	[below=of t0]				{}
			edge [entrata] (t0);
		\node	[transition] [label=left:$t_1$]				(t1)	[below=of p1,left=of p1]	{}
			edge [uscita,bend left] (p1)
			edge [entrata] (p1);
		\node	[transition] [label=right:$t_2$]				(t2)	[below=of p1,right=of p1]	{}
			edge [entrata] (p1);
		\node	[place]	[label=right:$p_2$]					(p2)	[below=of p1]				{}
			edge [entrata] (t1)
			edge [entrata] (t2);		
	\end{tikzpicture}
\end{center}
\end{frame}

\begin{frame}{A Simple Example}
\begin{center}
\begin{tikzpicture}
	% Styles
	[
		every label/.style={red},
		place/.style={circle,draw=blue!50,fill=blue!20,thick,inner sep=0,minimum size=6mm},
		transition/.style={rectangle,draw=black!50,fill=black!20,thick,inner sep=0,minimum size=4mm},
		bend angle=45,
		entrata/.style={<-,shorten <=1pt,>=stealth',thick},
		uscita/.style={->,shorten >=1pt,>=stealth',thick}
	]
	% Places
	\node[place,tokens=1] 		(waiting)									{};
	\node[place]		(critical)			[below=of waiting]		{};
	\node[place,tokens=2]		(semaphore)			[below=of critical,label=above:$s\le3$]		{};
	% Transitions
	\node[transition]			(leave critical)	[right=of critical]		{}
		edge 	[entrata]				(critical)
		edge 	[uscita,bend right]	node[auto,swap] {2}		(waiting)
		edge 	[entrata,bend left]		(semaphore);	
	\node[transition]	(enter critical)	[left=of critical]		{}
		edge [uscita]				(critical)
		edge [entrata,bend left]	(waiting)
		edge [uscita,bend right]	(semaphore);
	% Background
	\begin{pgfonlayer}{background}
		\node [fill=black!30,fit=(waiting) (critical) (semaphore) (leave critical) (enter critical)] {};
	\end{pgfonlayer}
\end{tikzpicture}
\end{center}
\end{frame}

